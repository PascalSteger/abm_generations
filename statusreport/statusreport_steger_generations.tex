% Created 2014-04-30 Wed 16:33
\documentclass[11pt]{article}
\usepackage[utf8]{inputenc}
\usepackage[T1]{fontenc}
\usepackage{fixltx2e}
\usepackage{graphicx}
\usepackage{longtable}
\usepackage{float}
\usepackage{wrapfig}
\usepackage{rotating}
\usepackage[normalem]{ulem}
\usepackage{amsmath}
\usepackage{textcomp}
\usepackage{marvosym}
\usepackage{wasysym}
\usepackage{amssymb}
\usepackage{hyperref}
\tolerance=1000
\author{Pascal Steger}
\date{\today}
\title{Generations in Bounded Confidence}
\hypersetup{
  pdfkeywords={},
  pdfsubject={},
  pdfcreator={Emacs 24.3.1 (Org mode 8.2.5h)}}
\begin{document}

\maketitle
\tableofcontents


\section{Abstract}
\label{sec-1}
I investigate the evolution and stability of an
agent based model for opinions in a group of agents with a
finite lifetime.
\section{Motivation}
\label{sec-2}
The bounded confidence model describes the way an ensemble of agents
converges on a common opinion, or several final opinions.

Effects that cannot be described with this model is long-time change
of opinions, as is e.g. seen in human societies over several
generations.

One way to mitigate this problem is by including said generations in the
model. I implement this idea by adding one additional internal
parameter for an agent, its age.

I propose as hypothesis that as old agents disappear to be
replaced by young, unbiased agents, a shift in the overall opinion
will occur. I will determine what model parameters have the most
influence and how the timescale for a considerable shift of opinion
compares to the lifetime of single agents.
\section{Implementation}
\label{sec-3}
I model agents $i\in[1,N]$ with two internal parameters: an opinion
$o_i$ and an age $a_i$.

I use several sets of initial conditions:

\begin{enumerate}
\item consensus: All agents have a common opinion $o_i=c\in[0,1]\,\forall
   i\in[1,N]$. The age is uniformly distributed. This setup enables us
to determine the timescale after which a consensus is lost due to
replacement of old agents with new agents having a uniform opinion
distribution.

\item uniform field: The opinions are uniformly distributed, and so are the
ages. This setup serves to determine the criteria for consensus
finding.
\end{enumerate}


I introduce a social distance $d$ between two agents encompassing
opinions $o_i$, physical positions $x_i$ (2D on a lattice, or 3
dimensional, or any appropriate social distance), age $a_i$. In each
timestep, two random agents meet. They interact based on two possible
rules:

\begin{enumerate}
\item if the distance between them is smaller than a parameter
   $\varepsilon$ with a strength parameter $\zeta$.

\item always, but the strength parameter is decreasing with increasing
distance: $\zeta=\zeta_0/g$.
\end{enumerate}

The interaction is based on the bounded confidence model

\begin{eqnarray}
    o_i(t+1) &=& o_i(t)+\zeta\cdot(o_j(t)-o_i(t))\\
    o_j(t+1) &=& o_j(t)+\zeta\cdot(o_i(t)-o_j(t))
\end{eqnarray}

\section{Preliminary Result}
\label{sec-4}
For the uniform field, and $\zeta=0.5$ with the distance-dependent
interaction rule, I get the evolution shown below. We can see the
initial convergence, and the appearance of new agents with time
(vertical lines mark births). A second "opinion peak" is forming at
timestep 800, with mostly young agents, but some old ones as well.
\includegraphics[width=.9\linewidth]{/home/psteger/abm14/generations/generations.png}
\section{Further Investigations}
\label{sec-5}
I will
\begin{itemize}
\item develop a metric to determine the number of concurrent stable opinions
\item plot the timescale between new opinions as a function of mean lifetime of
the agents divided by the frequency of new births
\item run a simulation starting from a consesus and compare it with the analytic treatment
\end{itemize}
% Emacs 24.3.1 (Org mode 8.2.5h)
\end{document}